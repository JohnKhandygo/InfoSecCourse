\documentclass[12pt]{article}

\usepackage[T2A]{fontenc}
\usepackage[utf8]{inputenc}
\usepackage[russian]{babel}

\usepackage{mathtext}
\usepackage{cmap}
\usepackage{amsmath,amssymb,amsthm,amscd,amsfonts}
\usepackage{graphicx}
\usepackage[colorbox,usenames,dvipsnames]{xcolor}
\usepackage{float, caption, subcaption, multirow}
\usepackage[section]{minted}
\usepackage{hyperref}

\voffset -24.5mm
\hoffset -5mm
\textwidth 173mm
\textheight 240mm
\oddsidemargin=0mm \evensidemargin=0mm

\definecolor{codegray}{gray}{0.9}
\newcommand{\code}[1]{\colorbox{codegray}{\texttt{#1}}}
\newmintinline[src]{console}{}

\renewcommand{\theFancyVerbLine}{\sffamily\textcolor{black}{\scriptsize\oldstylenums{\arabic{FancyVerbLine}}}}
\usemintedstyle[console]{bw}
\setminted[console] {
     bgcolor = white,
     linenos = false,
     tabsize = 2,
     formatcom = \color{black},
     frame = leftline,
     framerule = 0.8pt,
     framesep = 0.5cm,
     xleftmargin = 1cm,
     xbottommargin = 0.1cm
     breaklines = true,
     escapeinside=<>
}
\usemintedstyle[ruby]{tango}
\setminted[ruby] {
     bgcolor = white,
     linenos = true,
     tabsize = 2,
     formatcom = \color{black},
     frame = leftline,
     framerule = 0.8pt,
     framesep = 0.5cm,
     xleftmargin = 1cm,
     xbottommargin = 0.1cm
     breaklines = true,
     escapeinside=||
}

\begin{document}

\title{
    Лабораторная работа №3 \\ 
    Дисциплина <<Методы и средства защиты информации>> \\ 
    Metasploit}
\author{Евгений Хандыго, гр. 53501/3}

\maketitle
\tableofcontents

\newpage

\section{Постановка задачи}

В рамках данной работы необходимо овладеть основными аспектами работы с утилитой gpg4win. Ход работы соответствует следующим пунктам:
\begin{itemize}
    \item Изучить документацию gpg и gpg4win.
    \item Поставить собственную ЭЦП на файл.
    \item Проверить ЭЦП на стороннем файле.
    \item Обменяться зашифрованными сообщениями с другим пользователем gpg.
    \item Потренироваться в использовании утилиты gpg через интерфейс командной строки.
\end{itemize}
\section{Условия проведения экспериментов}

Ниже перечислены основные условия и настройка окружения, в котором проводились все описанные далее эксперименты:
\begin{itemize}
    \item В качестве атакуемой машины выступает виртуальная машина с Metasploitable.
    \item В качестве атакующей машины выступает виртуальная машина с Kali Linux.
    \item Атакуемая и атакующая машины связаны виртуальной сетью, в которую не входит никакая другая машина.
    \item Атакуемая и атакующая машины находятся в сети с адресом \code{10.0.0.0/24}. За атакуемой машиной закреплен ip-адресс
        \code{10.0.0.100}, а за атакующей --- \code{10.0.0.101}.
\end{itemize}
В рамках данной работы не будет рассматриваться этап поиска и сканирования атакуюемой машины со стороны атакуеющей, поскольку данный
процесс был подробно рассмотрен в предыдущей лабораторной работе. Таким образом, на атакующей машины уже существует workspace (в 
терминологии metasploit), в котором содержится вся необходимая для проведения атаки информации. Описание данного workspace приведено
в листинге \ref{lst:metasploit-db}

\inputminted[lastline=10]{console}{resources/environment/00_metasploit_db}
\inputminted[firstline=12]{console}{resources/environment/00_metasploit_db}
\captionof{listing}{Содержание workspace перед началом проведения экспериментов \label{lst:metasploit-db}}

% \begin{listing}[H]
%     \inputminted{console}{resources/environment/00_metasploit_db}
%     \caption{Содержание workspace перед началом проведения экспериментов}
%     \label{lst:metasploit-db}
% \end{listing}

Также стоит отметить, что некоторая несодержательная часть вывода утилит была удалена для удобства восприятия.
\section{Получение доступа к консоли через VNC-сервера}

Для получения доступа к VNC-серверу в общем случае необходимо получить пароль к нему. Конечно, прежде чем приступать к этому процессу,
необходимо проверить конфигурацию VNC-сервера на предмет отсутствия схемы авторизации. Данный факт указал бы на то, что получить
доступ к интерфейсу атакауемой машины можно без пароля. Для осущесвления этой проверки воспользуемся утилитой \code{vnc\_none\_auth} 
как показано в листинге \ref{lst:vnc-check-no-auth}.

\begin{listing}[H]
    \inputminted{console}{resources/vnc/00_check_no_auth}
    \caption{Проверка наличия схемы авторизации VNC-сервера}
    \label{lst:vnc-check-no-auth}
\end{listing}

Как видно из приведенного вывода, VNC-сервер, к которому необходимо получить доступ, не предоставляет свободного доступа. Таким образом,
необходимо каким-либо образом выяснить пароль к нему. Для этого в metasploit представлена утилита \code{vnc\_login}. По сути ее роль 
сводится к перебору заранее известного списка паролей. Ниже, в листинге \ref{lst:vnc-login-options} представлены параметры утилиты 
\code{vnc\_login}.
\\ \hfill \\
\inputminted[lastline=14]{console}{resources/vnc/01_login_options}
\inputminted[firstline=15]{console}{resources/vnc/01_login_options}
\captionof{listing}{Параметры утилиты \code{vnc\_login} \label{lst:vnc-login-options}}
% \begin{listing}[H]
%     \inputminted{console}{resources/vnc/01_login_options}
%     \caption{Параметры утилиты \code{vnc\_login}}
%     \label{lst:vnc-login-options}
% \end{listing}

Параметр с ключом \code{PASS\_FILE} указывает путь к файлу, содержащему возможные пароли для доступа к VNC-серверу на атакуемой машине.
Данный файл был создан заранее как показано в листинге \ref{lst:pswds-list}.

\begin{listing}[H]
    \inputminted{console}{resources/vnc/02_pswds_list}
    \caption{Создание списка возможных паролей к VNC-серверу}
    \label{lst:pswds-list}
\end{listing}

Перед тем, как запускать данную утилиту, необходимо также указать адресс атакуемой машины (или диапазон адресов). Пример использования
утилиты \code{vnc\_login} представлен в листинге \ref{lst:vnc-login-run}.

\begin{listing}[H]
    \inputminted{console}{resources/vnc/03_login_run}
    \caption{Пример вывода утилиты \code{vnc\_login}}
    \label{lst:vnc-login-run}
\end{listing}

Как видно из вывода, к VNC-серверу подошел один из паролей, представленных в файле \code{/home/pswds}. Используя его, теперь можно получить
доступ к атакуемой машине как показано в листинге \ref{lst:vnc-connect}.

\begin{listing}[H]
    \inputminted{console}{resources/vnc/04_vnc_connect}
    \caption{Инициализация соеденения с VNC-сервером по подобранному паролю}
    \label{lst:vnc-connect}
\end{listing}

Для подтверждения получения доступа к консоли атакуемой машины можно воспользоваться следующим набором команд:

\begin{listing}[H]
    \inputminted{console}{resources/vnc/05_vnc_test}
    \caption{Проверка присоединения к консоли атакуемой машины}
\end{listing}
\section{Получение списка публичных директорий \\ по протоколу SMB}

Для получения списка директорий в свободном доступе по протоколу SMB (\emph{Server Message Block}) metasploit предоставляет утилиту \code{smb\_enumshares}.
Пример использования данной утилиты представлен в листинге \ref{lst:smb}

\begin{listing}[H]
    \inputminted{console}{resources/smb/00_smb}
    \caption{Пример использования утилиты \code{smb\_enumshares}}
    \label{lst:smb}
\end{listing}
\section{Получение доступа к консоли через vsftpd}

Для эусплуатирования уязвимости типа backdoor в FTP-сервере vsftpd metasploit предоставляет утилиту \code{vsftpd\_234\_backdoor}. 
Ниже в листенге \ref{lst:vsftpd} приведен пример ее использования.

% \inputminted[lastline=7]{console}{resources/vsftpd/00_vsftpd}
% \inputminted[firstline=9]{console}{resources/vsftpd/00_vsftpd}
% \captionof{listing}{Пример использования утилиты \code{vsftpd\_234\_backdoor} \label{lst:vsftpd}}

\begin{listing}[H]
    \inputminted{console}{resources/vsftpd/00_vsftpd}
    \caption{Пример использования утилиты \code{vsftpd\_234\_backdoor}}
    \label{lst:vsftpd}
\end{listing}
\section{Получение доступа к консоли через irc}

Для эусплуатирования уязвимости типа backdoor в IRC-сервере UnrealIRCd metasploit предоставляет утилиту \code{unreal\_ircd\_3281\_backdoor}. 
Ниже в листенге \ref{lst:irc} приведен пример ее использования.

\begin{listing}[H]
    \inputminted{console}{resources/irc/00_irc}
    \caption{Пример использования утилиты \code{unreal\_ircd\_3281\_backdoor}}
    \label{lst:irc}
\end{listing}
\section{Разбор исходных кодов эксплойтов}

\subsection{Струткура эксплойтов}

Рассматриваемые в данном разделе примеры эксплойтов будут приведены на языке программирования Ruby. Все они имеют общую струткуру следующего вида:
\begin{itemize}
    \item Все эксплойты описываются одним классом, который расширяет класс \code{Msf::Exploit::Remote}.
    \item Каждый класс содержит поле \code{Rank}, описывающее эффективность данного эксплойта.
    \item Каждый класс содержит две обязательные функции:
    \begin{itemize}
        \item \code{initialize}, которая описывает метаинформацию об эксплойте и описывает его параметры.
        \item \code{exploit} функция, осуществляющая непосредственно эксплуатацию некоторой уязвимости.
    \end{itemize} 
\end{itemize}
Для удобства представления далее не будут рассматриваться секции инициализации эксплойтов, поскольку 
они описывают лишь служебную информацию программы. Ниже, в листинге \ref{lst:android-adb-info}, приведен пример реализации данной функции для модуля 
\code{Android ADB Debug Server Remote Payload Execution}.
% \\ \hfill \\
% \inputminted[firstline=15, lastline=35]{ruby}{resources/exploits/02_android_adb}
% \inputminted[firstline=37, lastline=42]{ruby}{resources/exploits/02_android_adb}
% \captionof{listing}{Пример реализацити функции \code{initialize} \label{lst:android-adb-info}}
\begin{listing}[H]
    \inputminted[firstline=15, lastline=42]{ruby}{resources/exploits/02_android_adb}
    \caption{Пример реализацити функции \code{initialize}}
    \label{lst:android-adb-info}
\end{listing}

\subsection{VSFTPD 2.3.4 - Backdoor Command Execution}

В данном разделе речь пойдет о модуле \code{vsftpd\_234\_backdoor}, который уже упоминался в данной работе. Ниже, приведен подробный разбор действий, необходимых для эксплуатации уязвимости. Код процедур, используемых 
в данном модуле представлен в листингах \ref{lst:vsftpd-exploit-src} и \ref{lst:vsftpd-handle-backdoor-src}.
\begin{itemize}
    \item \code{61 - 66}. Попытка открытия tcp-соеденения на порт \code{6200}. В случае успеха считается, что уязвимость уже эксплуатируется и управление передается процедуре \code{handle\_backdoor}.
    \item \code{68}. Попытка открытия tcp-соеденения на адресс и порт, указанные пользователем.
    \item \code{70 - 71}. Получение и печать приветственного сообщения от атакуемой машины.
    \item \code{73 - 75}. Отправление сообщения вида \code{USER username}, где \code{username} --- случайная строка, состоящая из цифр и букв длинной не более семи символов. Затем происходит считывание ответа и его 
        печать.
    \item \code{77 - 81}. Если ответ на предыдущее сообщение начинается с последовательности \code{530} (код ответа, соответствующий ситуации, когда пользователь не выполнил вход в систему), то эксплуатация 
        уязвимости считается невозможной. В этом случае происходит прерывание установленного соеденения и выход из процедуры.
    \item \code{83 - 87}. Если ответ на предыдущее сообщение не начинается с последовательности \code{331} (код ответа, соответствующий ситуации, когда имя пользователя корректно и для продолжения требуется пароль),
        то эксплуатация уязвимости считается невозможной. В этом случае происходит прерывание установленного соеденения и выход из процедуры.
    \item \code{89}. Отправление сообщения вида \code{PASS password}, где \code{password} --- случайная строка, состоящая из цифр и букв длинной не более семи символов.
    \item \code{92 - 97}. Попытка открытия tcp-соеденения на порт \code{6200}. В случае успеха управление передается процедуре \code{handle\_backdoor}.
    \item \code{99}. После выхода из процедуры \code{handle\_backdoor} происходит прерывание открытого соеденения и завершение процедуцры \code{exploit}.
\end{itemize}

\begin{listing}[H]
    \inputminted[firstline=60, lastline=100]{ruby}{resources/exploits/00_vsftpd}
    \caption{Реализацити процедуры \code{exploit} в модуле \code{vsftpd\_234\_backdoor}}
    \label{lst:vsftpd-exploit-src}
\end{listing}

\begin{listing}[H]
    \inputminted[firstline=102, lastline=116]{ruby}{resources/exploits/00_vsftpd}
    \caption{Реализацити процедуры \code{handle\_backdoor} в модуле \code{vsftpd\_234\_backdoor}}
    \label{lst:vsftpd-handle-backdoor-src}
\end{listing}

\subsection{Easy File Sharing HTTP Server 7.2 SEH Overflow}

В данном разделе будет рассмотрена реализация \code{SEH overflow exploit} для \code{Easy File Sharing HTTP Server} версии \code{7.2}. 

\begin{listing}[H]
    \inputminted[firstline=54, lastline=70]{ruby}{resources/exploits/01_easy_file_sharing}
    \caption{Реализацити процедуры \code{exploit} для \code{Easy File Sharing HTTP Server 7.2}}
    \label{lst:efshttps-exploit-src}
\end{listing}

Принцип работы данных уязвимостей основан на механизме \code{stuctured exception} \code{handling}, отвечающим за обработку программных исключений в операционной системе Windows. В SEH каждый блок обработки исключений 
ассоциирцется с собственным обработчиком исключений, который содержит указатель на следующий за ним (в иерархии) обработчик. Таким образом, при переполнении буффера, хранящего информацию о текущем обработчике 
исключений, можно перезаписать код следующего за ним обработчика на произвольный. Таким образом, передача управления в перезаписанный блок приведет к исполнению произвольного кода. 

Реализация процедуры, эксплуатирующей уязвимость данного типа для \code{Easy File Sharing HTTP Server} версии \code{7.2}, представлена в листинге \ref{lst:efshttps-exploit-src}. Ниже предсталвен разбор 
последовательности действий, выполняемых данной процедурой.
\begin{itemize}
    \item \code{55}. Попытка открытия tcp-соеденения на адресс и порт, указанные пользователем.
    \item \code{57}. Начало формирование строки, которая будет послана на сервер. В данном случае строка будет содержать слово \code{GET}, что скорее всего указывает на инструкцию использовать GET-запрос по протоколу
        HTTP.
    \item \code{58}. Строка-эксплуататор дополняется случайной строкой, состоящей из $4061$ой буквы в верхнем регистре. 
    \item \code{59}. Строка-эксплуататор дополняется структурой SEH-обработчика.
    \item \code{60}. Строка-эксплуататор дополняется строковым представлением пустой операции, повторенной $19$ раз.
    \item \code{61}. Строка-эксплуататор дополняется закодированной (видимо, для протокола HTTP) строкой полезной нагрузки --- кода, который необходимо выполнить на атакуемой машине.
    \item \code{62}. Строка-эксплуататор дополняется строковым представлением пустой операции, повторенной $7$ раз.
    \item \code{63 - 64}. Строка-эксплуататор дополняется случайной строкой, состоящей из букв в верхнем регистре остаточной длины. 
    \item \code{65}. Строка-эксплуататор дополняется строкой \code{HTTP/1.0}. 
    \item \code{66}. Строка-эксплуататор засылается на атакуемую машину.
    \item \code{68}. Проверка успеха эксплуатации и состояния соединения.
    \item \code{69}. Разрыв соеденения и выход из процедуры \code{exploit}.
\end{itemize}

\subsection{UnrealIRCD 3.2.8.1 - Backdoor Command Execution}

В данном разделе речь пойдет о модуле \code{unreal\_ircd\_3281\_backdoor}, который уже упоминался в данной работе. Ниже, приведен подробный разбор действий, необходимых для эксплуатации уязвимости. Код процедуры, 
\code{exploit} для данного модуля представлен в листинге \ref{lst:uircd-exploit-src}.

\begin{listing}[H]
    \inputminted[firstline=64, lastline=78]{ruby}{resources/exploits/03_irc}
    \caption{Реализацити процедуры \code{exploit} для модуля \code{unreal\_ircd\_3281\_backdoor}}
    \label{lst:uircd-exploit-src}
\end{listing}

\begin{itemize}
    \item \code{65}. Попытка открытия tcp-соеденения на адресс и порт, указанные пользователем.
    \item \code{67 - 71}. Печать приветственного сообщения.
    \item \code{74}. Отправление сообщения-эксплуататора на атакуемую машину.
    \item \code{76}. Проверка успеха эксплуатации и состояния соединения.
    \item \code{77}. Разрыв соеденения и выход из процедуры \code{exploit}.
\end{itemize}
\section{Заключение}

Фреймворк metasploit является очень мощным инструментом для проведения тестирования на проникновение. В рамках данной работы были освоены некоторые базовые аспекты работы с metasploit, а также проведена проверка его
возможности на практике путем эксплуатирования нескольких заведомо известных уязвимостей атакуемой машины.

\end{document}