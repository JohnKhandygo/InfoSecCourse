\section{Условия проведения экспериментов}

Ниже перечислены основные условия и настройка окружения, в котором проводились все описанные далее эксперименты:
\begin{itemize}
    \item В качестве атакуемой машины выступает виртуальная машина с Metasploitable.
    \item В качестве атакующей машины выступает виртуальная машина с Kali Linux.
    \item Атакуемая и атакующая машины связаны виртуальной сетью, в которую не входит никакая другая машина.
    \item Атакуемая и атакующая машины находятся в сети с адресом \code{10.0.0.0/24}. За атакуемой машиной закреплен ip-адресс
        \code{10.0.0.100}, а за атакующей --- \code{10.0.0.101}.
\end{itemize}
В рамках данной работы не будет рассматриваться этап поиска и сканирования атакуюемой машины со стороны атакуеющей, поскольку данный
процесс был подробно рассмотрен в предыдущей лабораторной работе. Таким образом, на атакующей машины уже существует workspace (в 
терминологии metasploit), в котором содержится вся необходимая для проведения атаки информации. Описание данного workspace приведено
в листинге \ref{lst:metasploit-db}

\inputminted[lastline=10]{console}{resources/environment/00_metasploit_db}
\inputminted[firstline=12]{console}{resources/environment/00_metasploit_db}
\captionof{listing}{Содержание workspace перед началом проведения экспериментов \label{lst:metasploit-db}}

% \begin{listing}[H]
%     \inputminted{console}{resources/environment/00_metasploit_db}
%     \caption{Содержание workspace перед началом проведения экспериментов}
%     \label{lst:metasploit-db}
% \end{listing}

Также стоит отметить, что некоторая несодержательная часть вывода утилит была удалена для удобства восприятия.