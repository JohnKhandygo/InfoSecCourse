\section{Разбор исходных кодов эксплойтов}

\subsection{Струткура эксплойтов}

Рассматриваемые в данном разделе примеры эксплойтов будут приведены на языке программирования Ruby. Все они имеют общую струткуру следующего вида:
\begin{itemize}
    \item Все эксплойты описываются одним классом, который расширяет класс \code{Msf::Exploit::Remote}.
    \item Каждый класс содержит поле \code{Rank}, описывающее эффективность данного эксплойта.
    \item Каждый класс содержит две обязательные функции:
    \begin{itemize}
        \item \code{initialize}, которая описывает метаинформацию об эксплойте и описывает его параметры.
        \item \code{exploit} функция, осуществляющая непосредственно эксплуатацию некоторой уязвимости.
    \end{itemize} 
\end{itemize}
Для удобства представления далее не будут рассматриваться секции инициализации эксплойтов, поскольку 
они описывают лишь служебную информацию программы. Ниже, в листинге \ref{lst:android-adb-info}, приведен пример реализации данной функции для модуля 
\code{Android ADB Debug Server Remote Payload Execution}.
% \\ \hfill \\
% \inputminted[firstline=15, lastline=35]{ruby}{resources/exploits/02_android_adb}
% \inputminted[firstline=37, lastline=42]{ruby}{resources/exploits/02_android_adb}
% \captionof{listing}{Пример реализацити функции \code{initialize} \label{lst:android-adb-info}}
\begin{listing}[H]
    \inputminted[firstline=15, lastline=42]{ruby}{resources/exploits/02_android_adb}
    \caption{Пример реализацити функции \code{initialize}}
    \label{lst:android-adb-info}
\end{listing}

\subsection{VSFTPD 2.3.4 - Backdoor Command Execution}

В данном разделе речь пойдет о модуле \code{vsftpd\_234\_backdoor}, который уже упоминался в данной работе. Ниже, приведен подробный разбор действий, необходимых для эксплуатации уязвимости. Код процедур, используемых 
в данном модуле представлен в листингах \ref{lst:vsftpd-exploit-src} и \ref{lst:vsftpd-handle-backdoor-src}.
\begin{itemize}
    \item \code{61 - 66}. Попытка открытия tcp-соеденения на порт \code{6200}. В случае успеха считается, что уязвимость уже эксплуатируется и управление передается процедуре \code{handle\_backdoor}.
    \item \code{68}. Попытка открытия tcp-соеденения на адресс и порт, указанные пользователем.
    \item \code{70 - 71}. Получение и печать приветственного сообщения от атакуемой машины.
    \item \code{73 - 75}. Отправление сообщения вида \code{USER username}, где \code{username} --- случайная строка, состоящая из цифр и букв длинной не более семи символов. Затем происходит считывание ответа и его 
        печать.
    \item \code{77 - 81}. Если ответ на предыдущее сообщение начинается с последовательности \code{530} (код ответа, соответствующий ситуации, когда пользователь не выполнил вход в систему), то эксплуатация 
        уязвимости считается невозможной. В этом случае происходит прерывание установленного соеденения и выход из процедуры.
    \item \code{83 - 87}. Если ответ на предыдущее сообщение не начинается с последовательности \code{331} (код ответа, соответствующий ситуации, когда имя пользователя корректно и для продолжения требуется пароль),
        то эксплуатация уязвимости считается невозможной. В этом случае происходит прерывание установленного соеденения и выход из процедуры.
    \item \code{89}. Отправление сообщения вида \code{PASS password}, где \code{password} --- случайная строка, состоящая из цифр и букв длинной не более семи символов.
    \item \code{92 - 97}. Попытка открытия tcp-соеденения на порт \code{6200}. В случае успеха управление передается процедуре \code{handle\_backdoor}.
    \item \code{99}. После выхода из процедуры \code{handle\_backdoor} происходит прерывание открытого соеденения и завершение процедуцры \code{exploit}.
\end{itemize}

\begin{listing}[H]
    \inputminted[firstline=60, lastline=100]{ruby}{resources/exploits/00_vsftpd}
    \caption{Реализацити процедуры \code{exploit} в модуле \code{vsftpd\_234\_backdoor}}
    \label{lst:vsftpd-exploit-src}
\end{listing}

\begin{listing}[H]
    \inputminted[firstline=102, lastline=116]{ruby}{resources/exploits/00_vsftpd}
    \caption{Реализацити процедуры \code{handle\_backdoor} в модуле \code{vsftpd\_234\_backdoor}}
    \label{lst:vsftpd-handle-backdoor-src}
\end{listing}

\subsection{Easy File Sharing HTTP Server 7.2 SEH Overflow}

В данном разделе будет рассмотрена реализация \code{SEH overflow exploit} для \code{Easy File Sharing HTTP Server} версии \code{7.2}. 

\begin{listing}[H]
    \inputminted[firstline=54, lastline=70]{ruby}{resources/exploits/01_easy_file_sharing}
    \caption{Реализацити процедуры \code{exploit} для \code{Easy File Sharing HTTP Server 7.2}}
    \label{lst:efshttps-exploit-src}
\end{listing}

Принцип работы данных уязвимостей основан на механизме \code{stuctured exception} \code{handling}, отвечающим за обработку программных исключений в операционной системе Windows. В SEH каждый блок обработки исключений 
ассоциирцется с собственным обработчиком исключений, который содержит указатель на следующий за ним (в иерархии) обработчик. Таким образом, при переполнении буффера, хранящего информацию о текущем обработчике 
исключений, можно перезаписать код следующего за ним обработчика на произвольный. Таким образом, передача управления в перезаписанный блок приведет к исполнению произвольного кода. 

Реализация процедуры, эксплуатирующей уязвимость данного типа для \code{Easy File Sharing HTTP Server} версии \code{7.2}, представлена в листинге \ref{lst:efshttps-exploit-src}. Ниже предсталвен разбор 
последовательности действий, выполняемых данной процедурой.
\begin{itemize}
    \item \code{55}. Попытка открытия tcp-соеденения на адресс и порт, указанные пользователем.
    \item \code{57}. Начало формирование строки, которая будет послана на сервер. В данном случае строка будет содержать слово \code{GET}, что скорее всего указывает на инструкцию использовать GET-запрос по протоколу
        HTTP.
    \item \code{58}. Строка-эксплуататор дополняется случайной строкой, состоящей из $4061$ой буквы в верхнем регистре. 
    \item \code{59}. Строка-эксплуататор дополняется структурой SEH-обработчика.
    \item \code{60}. Строка-эксплуататор дополняется строковым представлением пустой операции, повторенной $19$ раз.
    \item \code{61}. Строка-эксплуататор дополняется закодированной (видимо, для протокола HTTP) строкой полезной нагрузки --- кода, который необходимо выполнить на атакуемой машине.
    \item \code{62}. Строка-эксплуататор дополняется строковым представлением пустой операции, повторенной $7$ раз.
    \item \code{63 - 64}. Строка-эксплуататор дополняется случайной строкой, состоящей из букв в верхнем регистре остаточной длины. 
    \item \code{65}. Строка-эксплуататор дополняется строкой \code{HTTP/1.0}. 
    \item \code{66}. Строка-эксплуататор засылается на атакуемую машину.
    \item \code{68}. Проверка успеха эксплуатации и состояния соединения.
    \item \code{69}. Разрыв соеденения и выход из процедуры \code{exploit}.
\end{itemize}

\subsection{UnrealIRCD 3.2.8.1 - Backdoor Command Execution}

В данном разделе речь пойдет о модуле \code{unreal\_ircd\_3281\_backdoor}, который уже упоминался в данной работе. Ниже, приведен подробный разбор действий, необходимых для эксплуатации уязвимости. Код процедуры, 
\code{exploit} для данного модуля представлен в листинге \ref{lst:uircd-exploit-src}.

\begin{listing}[H]
    \inputminted[firstline=64, lastline=78]{ruby}{resources/exploits/03_irc}
    \caption{Реализацити процедуры \code{exploit} для модуля \code{unreal\_ircd\_3281\_backdoor}}
    \label{lst:uircd-exploit-src}
\end{listing}

\begin{itemize}
    \item \code{65}. Попытка открытия tcp-соеденения на адресс и порт, указанные пользователем.
    \item \code{67 - 71}. Печать приветственного сообщения.
    \item \code{74}. Отправление сообщения-эксплуататора на атакуемую машину.
    \item \code{76}. Проверка успеха эксплуатации и состояния соединения.
    \item \code{77}. Разрыв соеденения и выход из процедуры \code{exploit}.
\end{itemize}