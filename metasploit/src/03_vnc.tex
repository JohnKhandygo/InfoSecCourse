\section{Получение доступа к консоли через VNC-сервера}

Для получения доступа к VNC-серверу в общем случае необходимо получить пароль к нему. Конечно, прежде чем приступать к этому процессу,
необходимо проверить конфигурацию VNC-сервера на предмет отсутствия схемы авторизации. Данный факт указал бы на то, что получить
доступ к интерфейсу атакауемой машины можно без пароля. Для осущесвления этой проверки воспользуемся утилитой \code{vnc\_none\_auth} 
как показано в листинге \ref{lst:vnc-check-no-auth}.

\begin{listing}[H]
    \inputminted{console}{resources/vnc/00_check_no_auth}
    \caption{Проверка наличия схемы авторизации VNC-сервера}
    \label{lst:vnc-check-no-auth}
\end{listing}

Как видно из приведенного вывода, VNC-сервер, к которому необходимо получить доступ, не предоставляет свободного доступа. Таким образом,
необходимо каким-либо образом выяснить пароль к нему. Для этого в metasploit представлена утилита \code{vnc\_login}. По сути ее роль 
сводится к перебору заранее известного списка паролей. Ниже, в листинге \ref{lst:vnc-login-options} представлены параметры утилиты 
\code{vnc\_login}.
\\ \hfill \\
\inputminted[lastline=14]{console}{resources/vnc/01_login_options}
\inputminted[firstline=15]{console}{resources/vnc/01_login_options}
\captionof{listing}{Параметры утилиты \code{vnc\_login} \label{lst:vnc-login-options}}
% \begin{listing}[H]
%     \inputminted{console}{resources/vnc/01_login_options}
%     \caption{Параметры утилиты \code{vnc\_login}}
%     \label{lst:vnc-login-options}
% \end{listing}

Параметр с ключом \code{PASS\_FILE} указывает путь к файлу, содержащему возможные пароли для доступа к VNC-серверу на атакуемой машине.
Данный файл был создан заранее как показано в листинге \ref{lst:pswds-list}.

\begin{listing}[H]
    \inputminted{console}{resources/vnc/02_pswds_list}
    \caption{Создание списка возможных паролей к VNC-серверу}
    \label{lst:pswds-list}
\end{listing}

Перед тем, как запускать данную утилиту, необходимо также указать адресс атакуемой машины (или диапазон адресов). Пример использования
утилиты \code{vnc\_login} представлен в листинге \ref{lst:vnc-login-run}.

\begin{listing}[H]
    \inputminted{console}{resources/vnc/03_login_run}
    \caption{Пример вывода утилиты \code{vnc\_login}}
    \label{lst:vnc-login-run}
\end{listing}

Как видно из вывода, к VNC-серверу подошел один из паролей, представленных в файле \code{/home/pswds}. Используя его, теперь можно получить
доступ к атакуемой машине как показано в листинге \ref{lst:vnc-connect}.

\begin{listing}[H]
    \inputminted{console}{resources/vnc/04_vnc_connect}
    \caption{Инициализация соеденения с VNC-сервером по подобранному паролю}
    \label{lst:vnc-connect}
\end{listing}

Для подтверждения получения доступа к консоли атакуемой машины можно воспользоваться следующим набором команд:

\begin{listing}[H]
    \inputminted{console}{resources/vnc/05_vnc_test}
    \caption{Проверка присоединения к консоли атакуемой машины}
\end{listing}