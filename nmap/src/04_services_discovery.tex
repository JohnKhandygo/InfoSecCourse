\section{Определение версии сервисов}

Nmap производит определение типа сервиса, слушающего некоторый порт на основании информации в файле \src{nmap-services}. По сути этот
файл просто описывает наиболее вероятный порт для всякого известного ему сервиса. Для более точного определения того, какая служба
запущена на конкретном сервисе используется опция \src{-sV}. Здесь уже работают более сложные правила, которые предполагают посылку
запросов определнного вида и сравнение полученных ответов с некоторыми шаблонами. Пример использования опции \src{-sV} представлен
в листинге \ref{lst:scan-ports-sV}.

\begin{listing}[H]
    \inputminted{console}{resources/06_scan_ports_sV}
    \caption{Подробное изучение сервисов с использованием опции \src{-sV}}
    \label{lst:scan-ports-sV}
\end{listing}

Правила, по которым определяются слуюбы и их версии расположены в файле \\ \src{nmap-service-probes}. Для того, чтобы продемонстрировать
его работу создадим собственный минимальный TCP сервер и попробуем добавить соответствующее правило для его распознавания. Исходный код 
сервера представлен в листинге \ref{lst:my-tcp-server}.

\begin{listing}[H]
    \inputminted[linenos=true]{console}{resources/07_my_tcp_server}
    \caption{Исходный код минимального TCP сервера}
    \label{lst:my-tcp-server}
\end{listing}

Механизм работы данной службы тривиален:
\begin{itemize}
    \item В строке $7$ создает сокет для прослушки TCP запросов на порт с номером $6789$.
    \item Внутри бесконечного цикла в строке $9$ происходит ожидание сообщений на прослушиваемый порт.
    \item В строке $14$ в ответ помещается сообщение \src{"Hello, nice to meet you."}.
\end{itemize}
Для запуска данного сервера необходимо поместить данный код в файл \src{TCPServer.java} и разместить его на сканируемой машине. 
Затем данный файл необходимо скомпилировать. Для этого используем команду \src{javac TCPServer.java}. Для запуска полученного таким 
образом \src{.class} файла достаточно выполнить команду \src{java TCPServer}.

Теперь для того, чтобы <<научить>> Nmap распознавать созданную службу необходимо определить какого типа запрос должен быть направлен
в даннуб службу для того, чтобы получить ожидаемый и, по возможности уникальный в пределах остальных известных Nmap сервисов, ответ.
Понятно, что, поскольку ответ, выдаваемый созданным сервисом не зависит от направленного ему запроса, можно использовать любой
TCP запрос, например, запрос с пустым телом. Ответ сервера также не зависит ни от каких условий и будет содержать всегда одну и ту
же строку. Таким, образом, сегмент файла \src{nmap-service-probes}, который отвечает за распознавание написанного TCP сервера,
может выглядеть как показано в листинге \ref{lst:my-tcp-server-probe}.

\begin{listing}[H]
    \inputminted{console}{resources/08_nmap_service_probes}
    \caption{Сегмент файла \src{nmap-service-probes}, определяющий \src{my-tcp}}
    \label{lst:my-tcp-server-probe}
\end{listing}

Для того чтобы проверить корректность определения созданного сервера, воспользуемся уже известной опцией сканирования сервисов с
указанием порта как показано в листинге \ref{lst:scan-port-6789}.

\begin{listing}[H]
    \inputminted{console}{resources/09_scan_port_6789}
    \caption{Пример сканирования порта с номером \src{6789}}
    \label{lst:scan-port-6789}
\end{listing}