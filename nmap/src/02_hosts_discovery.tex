\section{Поиск активных хостов}

При вызове утилиты Nmap без дополнительных параметров поиск активных хостов в заданной сети будет произведен автоматически. Недостаток
такого подхода заключается в том, что вместе с этим будут также просканированы порты обнаруженных машин, что может привлечь излишнее 
внимание к хосту, инициализироваашему сканирование. К тому же, при наличии достаточно большого количества активных машин в сети, такая
операция может занять ощутимо большое количество времени. Таким образом при поиске активных хостов целесообразно избежать этапа 
сканирования портов. Для этого можно воспользоваться флагом \src{-sn} как показано в листинге \ref{lst:scan-hosts-def}. 

\begin{listing}[H]
    \inputminted{console}{resources/01_scan_hosts_default}
    \caption{Поиск активных хостов с использованием Nmap}
    \label{lst:scan-hosts-def}
\end{listing}

Здесь пр вызове утилиты Nmap также была использована опция \src{-n}, которая указывает Nmap не пытаться делать обратное разрешение 
адресов (reserve DNS resolution) обнаруженных активных машин. В данном случае использование этой опции можно считать полностью 
оправданным поскольку в сканируемой сети отсутствует DNS сервер. В реальной жизни DNS-имя хоста в сети потенциально может о многом
сказать, однако, стоит помнить, что процесс обратного разрешения ip адресов может существенно замедлить проесс сканирования 
\footnote{https://nmap.org/book/man-host-discovery.html (see \src{-n} section)}. Таким образом, вызов Nmap, приведенный в листинге 
\ref{lst:scan-hosts-def} соответствует операции поиска активных машин в сети с адресом \src{10.0.0.0} с маской \src{255.255.255.0} без
совершения обратного разрешения ip адресов обнаруженных хостов. 

Nmap предоставляет несколько опций для обнаружения активных хостов. Наиболее быстрой считается опция \src{-PR}. Данный флаг считается 
значением по умолчанию и, более того, будет использован в случаях, когда явно указан какой-либо другой флаг. Опция \src{-PR} 
соответствует широкофещательному запросу в протоколе ARP. Конечно, использование данного протокола не всегда возможно. Для 
принудительной отмены использования ARP можно использовать флаг \src{--disable-arp-ping}. В качестве примера рассмотрим опцию 
\src{-PS}, которая соответствует отправке TCP запроса (по умолчанию на порт \src{80}) с флагом \src{SYN}. Вне зависимости от ответа 
сканируемого хоста Nmap прервет процесс \emph{тройного рукопожатия}. Суть данного метода заключается в том, что ответ порта в сущности
не имеет значения, важен факт наличия ответа. Пример использования опции \src{--PS} представлен в листинге \ref{lst:scan-hosts-PS}.

\begin{listing}[H]
    \inputminted{console}{resources/02_scan_hosts_PS}
    \caption{Поиск активных хостов с использованием Nmap и опцией \src{-PS}}
    \label{lst:scan-hosts-PS}
\end{listing}