\section{Использование db\_nmap}

Утилита db\_nmap входит в состав фреймворка metasploit. Данный инструмент расширяет возможности Nmap тем, что позволяет сохранять 
результаты сканирования в базу данных. Для того, чтобы начать работу с db\_nmap необходимо сначала запустить сервис базы данных. В 
случае дистрибутива kali linux это будет PostgreSQL. Для запуска сервиса базы данных воспользуемся командой 
\src{service postgresql start}. Для работы непосредственно с db\_nmap теперь необходимо открыть консольный интерфейс для metasploit. 
Для удобной работы с базой данных заведем новый workspace, который будет хранить результаты для сканирования хоста по адресу 
\src{10.0.0.100} как показано в листинге \ref{lst:workspace-a}.

\begin{listing}[H]
    \inputminted{console}{resources/14_workspace_a}
    \caption{Создание нового workspace для работы с db\_nmap}
    \label{lst:workspace-a}
\end{listing}

Синтаксис утилиты db\_nmap повторяет синтаксис вызова Nmap. Для демонстрации работы с db\_nmap проведем сканирование TCP портов хоста
с адресом \src{10.0.0.100} как показано в листинге \ref{lst:db-scan-ports-sV}.

\begin{listing}[H]
    \inputminted{console}{resources/15_db_scan_ports_sV}
    \caption{Пример использования утилиты db\_nmap}
    \label{lst:db-scan-ports-sV}
\end{listing}

Теперь для проверки результатов работы db\_nmap воспользуемся командами \src{hosts} и \src{services} как показано в листингах 
\ref{lst:db-hosts} и \ref{lst:db-services}.

\begin{listing}[H]
    \inputminted{console}{resources/16_db_scan_hosts}
    \caption{Результаты сканирования db\_nmap}
    \label{lst:db-hosts}
\end{listing}

\begin{listing}[H]
    \inputminted{console}{resources/17_db_services}
    \caption{Результаты сканирования db\_nmap}
    \label{lst:db-services}
\end{listing}