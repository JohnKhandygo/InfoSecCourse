\section{Анализ работы Nmap}

Для того, чтобы проанализировать работу Nmap воспользуемся утилитой Wireshark. 

Рассмотрим задачу поиска активных хостов в сети. Для удобства предположим, что заранее известен не очень широкий диапазон потенциально
уязвимых адресов и возможно применение опции \src{-PR}. Таким образом, для решения поставленной задачи можно воспользоваться командой
\begin{center}
    \src{nmap -n -sn 10.0.0.98-102}.
\end{center}
Результаты анализа трафика данной команды с помощью инструмента Wireshark представлены в литиснге \ref{lst:wshark-PR}.

\begin{listing}[H]
    \inputminted[linenos=true]{console}{resources/12_wshark_PR}
    \caption{Анализ трафика для \src{nmap -n -sn 10.0.0.98-102}}
    \label{lst:wshark-PR}
\end{listing}

Как видно из отчета, процесс поиска активных хостов по протоколу ARP достаточно прост: Nmap просто рассылает широковещательный запрос
для того, чтобы узнат, есть ли в сети машина с определенным ip адресом. В случае получения ответа (строка 10) хост считается активным.

Проанализируем теперь сканирование известного сервиса на порту \src{6789}. Для этого воспользуемся командой 
\begin{center}
    \src{nmap -n -sV -p6789 10.0.0.100}.
\end{center}
Результаты анализа приведены в листинге \ref{lst:wshark-P6789}. Провдем подробный разбор:
\begin{itemize}
    \item В строках $2 -- 9$ отражена проверка активности хоста по сценарию, аналогичному \ref{lst:wshark-PR}.  
    \item В строках $10 -- 12$ происходит проверка того, что порт открыт.
    \item В строках $13 -- 15$ происходит процесс тройного рукопожатия и установление соеденения.
    \item В строках $15 -- 23$ происходит обмен сообщениями и завершение соеденения.
\end{itemize}

\begin{listing}[H]
    \inputminted[linenos=true]{console}{resources/13_wshark_P6789}
    \caption{Анализ трафика для \src{nmap -n -sV -p6789 10.0.0.100}}
    \label{lst:wshark-P6789}
\end{listing}