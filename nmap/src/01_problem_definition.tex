\section{Постановка задачи}

В рамках данной работы необходимо овладеть основными аспектами работы с утилитой Nmap. Ход работы соответствует следующим пунктам:
\begin{itemize}
    \item Провести поиск активных хостов.
    \item Определить открытые порты.
    \item Определить версии сервисов.
    \item Изучить файлы nmap-services, nmap-os-db, nmap-service-probes
    \item Добавить новую сигнатуру службы в файл nmap-service-probes.
    \item Сохранить вывод утилиты в формате xml.
    \item Исследвать различные этапы и режимы работы Nmap с использованием утилиты Wireshark.
    \item Просканировать виртуальную машину Metasploitable2 используя утилиту db\_nmap.
    \item Выбрать пять записей из файла nmap-service-probes и описать их работу.
    \item Выбрать один скрипт из состава Nmap и описать его работу.
\end{itemize}
\emph{Nmap} (\emph{Network Mapper}) --- это утилита предназначенная для изучения состояния и проверки безопасности компьютерных сетей. 
Общий синтаксис вызова даннной утилиты приведен в листинге \ref{lst:nmap-syntax}.

\begin{listing}[H]
    \inputminted{console}{resources/00_nmap_syntax}
    \caption{Синтаксис вызова утилиты Nmap}
    \label{lst:nmap-syntax}
\end{listing}