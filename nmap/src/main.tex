\documentclass[12pt]{article}

\usepackage[T2A]{fontenc}
\usepackage[utf8]{inputenc}
\usepackage[russian]{babel}

\usepackage{mathtext}
\usepackage{cmap}
\usepackage{amsmath,amssymb,amsthm,amscd,amsfonts}
\usepackage{graphicx}
\usepackage[colorbox,usenames,dvipsnames]{xcolor}
\usepackage{float, caption, subcaption, multirow}
\usepackage[section]{minted}
\usepackage{hyperref}

\voffset -24.5mm
\hoffset -5mm
\textwidth 173mm
\textheight 240mm
\oddsidemargin=0mm \evensidemargin=0mm

\definecolor{codegray}{gray}{0.9}
\newcommand{\code}[1]{\colorbox{codegray}{\texttt{#1}}}
\newmintinline[src]{console}{}

\renewcommand{\theFancyVerbLine}{\sffamily\textcolor{black}{\scriptsize\oldstylenums{\arabic{FancyVerbLine}}}}
\usemintedstyle[console]{bw}
\setminted[console] {
     bgcolor = codegray,
     linenos = false,
     tabsize = 2,
     formatcom = \color{black},
     frame = leftline,
     framerule = 0.8pt,
     framesep = 0.5cm,
     xleftmargin = 1cm,
     xbottommargin = 0.1cm
     breaklines = true,
     escapeinside=<>
}
\usemintedstyle[xml]{emacs}
\setminted[xml] {
     bgcolor = codegray,
     linenos = true,
     tabsize = 2,
     formatcom = \color{black},
     frame = leftline,
     framerule = 0.8pt,
     framesep = 0.5cm,
     xleftmargin = 1cm,
     xbottommargin = 0.1cm
     breaklines = true,
     escapeinside=||
}

\begin{document}

\title{
    Лабораторная работа №2 \\ 
    Дисциплина <<Методы и средства защиты информации>> \\ 
    Nmap}
\author{Евгений Хандыго, гр. 53501/3}

\maketitle
\tableofcontents

\newpage

\section{Постановка задачи}

В рамках данной работы необходимо овладеть основными аспектами работы с утилитой gpg4win. Ход работы соответствует следующим пунктам:
\begin{itemize}
    \item Изучить документацию gpg и gpg4win.
    \item Поставить собственную ЭЦП на файл.
    \item Проверить ЭЦП на стороннем файле.
    \item Обменяться зашифрованными сообщениями с другим пользователем gpg.
    \item Потренироваться в использовании утилиты gpg через интерфейс командной строки.
\end{itemize}
\section{Поиск активных хостов}

При вызове утилиты Nmap без дополнительных параметров поиск активных хостов в заданной сети будет произведен автоматически. Недостаток
такого подхода заключается в том, что вместе с этим будут также просканированы порты обнаруженных машин, что может привлечь излишнее 
внимание к хосту, инициализироваашему сканирование. К тому же, при наличии достаточно большого количества активных машин в сети, такая
операция может занять ощутимо большое количество времени. Таким образом при поиске активных хостов целесообразно избежать этапа 
сканирования портов. Для этого можно воспользоваться флагом \src{-sn} как показано в листинге \ref{lst:scan-hosts-def}. 

\begin{listing}[H]
    \inputminted{console}{resources/01_scan_hosts_default}
    \caption{Поиск активных хостов с использованием Nmap}
    \label{lst:scan-hosts-def}
\end{listing}

Здесь пр вызове утилиты Nmap также была использована опция \src{-n}, которая указывает Nmap не пытаться делать обратное разрешение 
адресов (reserve DNS resolution) обнаруженных активных машин. В данном случае использование этой опции можно считать полностью 
оправданным поскольку в сканируемой сети отсутствует DNS сервер. В реальной жизни DNS-имя хоста в сети потенциально может о многом
сказать, однако, стоит помнить, что процесс обратного разрешения ip адресов может существенно замедлить проесс сканирования 
\footnote{https://nmap.org/book/man-host-discovery.html (see \src{-n} section)}. Таким образом, вызов Nmap, приведенный в листинге 
\ref{lst:scan-hosts-def} соответствует операции поиска активных машин в сети с адресом \src{10.0.0.0} с маской \src{255.255.255.0} без
совершения обратного разрешения ip адресов обнаруженных хостов. 

Nmap предоставляет несколько опций для обнаружения активных хостов. Наиболее быстрой считается опция \src{-PR}. Данный флаг считается 
значением по умолчанию и, более того, будет использован в случаях, когда явно указан какой-либо другой флаг. Опция \src{-PR} 
соответствует широкофещательному запросу в протоколе ARP. Конечно, использование данного протокола не всегда возможно. Для 
принудительной отмены использования ARP можно использовать флаг \src{--disable-arp-ping}. В качестве примера рассмотрим опцию 
\src{-PS}, которая соответствует отправке TCP запроса (по умолчанию на порт \src{80}) с флагом \src{SYN}. Вне зависимости от ответа 
сканируемого хоста Nmap прервет процесс \emph{тройного рукопожатия}. Суть данного метода заключается в том, что ответ порта в сущности
не имеет значения, важен факт наличия ответа. Пример использования опции \src{--PS} представлен в листинге \ref{lst:scan-hosts-PS}.

\begin{listing}[H]
    \inputminted{console}{resources/02_scan_hosts_PS}
    \caption{Поиск активных хостов с использованием Nmap и опцией \src{-PS}}
    \label{lst:scan-hosts-PS}
\end{listing}
\section{Поиск открытых портов}

При обнаружении открытых портов Nmap также предоставляет несколько опций. По умолчанию используется флаг \src{-sS}, который 
соответствует посылке TCP запроса с флагом \src{SYN}. Преимущество такого подхода заключается в том, что в ходе сканирования не 
устанавливается никаких соединений. Пример сканирования портов утилитой Nmap представлен в листинге \ref{lst:scan-ports-sS}. 

\begin{listing}[H]
    \inputminted{console}{resources/03_scan_ports_sS}
    \caption{Поиск активных портов с использованием Nmap и опции \src{-sS}}
    \label{lst:scan-ports-sS}
\end{listing}

Конечно, использование опции \src{-sS} оправдано только для сканирования портов, на которых запущены службы, способные работать с 
протоколом TCP. Для обнаружения портов, которые, например, настроены на прием UDP пакетов необходимо воспользоваться опцией \src{-sU}
как показано в листинге \ref{lst:scan-ports-sU}.

\begin{listing}[H]
    \inputminted{console}{resources/04_scan_ports_sU}
    \caption{Поиск активных портов с использованием Nmap и опции \src{-sU}}
    \label{lst:scan-ports-sU}
\end{listing}

Еще одной интересной опцией, предоставляемой Nmap для сканирования портов, является флаг \src{-sA}. Функционирование Nmap при 
использовании данной опции заключается в отправке TCP запроса с флагом \src{ACK}. В таком случае утилита не способна определит открыт
или закрыт порт, однако способна определить какие из них отфильтрованы (то есть закрыты брандмауэром (firewall)). Такая функция может
оказаться полезной, например, при проведении аудита безопасни. Пример использования флага \src{-sA} представлен в листинге 
\ref{lst:scan-ports-sA}.

\begin{listing}[H]
    \inputminted{console}{resources/05_scan_ports_sA}
    \caption{Поиск активных портов с использованием Nmap и опции \src{-sA}}
    \label{lst:scan-ports-sA}
\end{listing}
\section{Определение версии сервисов}

Nmap производит определение типа сервиса, слушающего некоторый порт на основании информации в файле \src{nmap-services}. По сути этот
файл просто описывает наиболее вероятный порт для всякого известного ему сервиса. Для более точного определения того, какая служба
запущена на конкретном сервисе используется опция \src{-sV}. Здесь уже работают более сложные правила, которые предполагают посылку
запросов определнного вида и сравнение полученных ответов с некоторыми шаблонами. Пример использования опции \src{-sV} представлен
в листинге \ref{lst:scan-ports-sV}.

\begin{listing}[H]
    \inputminted{console}{resources/06_scan_ports_sV}
    \caption{Подробное изучение сервисов с использованием опции \src{-sV}}
    \label{lst:scan-ports-sV}
\end{listing}

Правила, по которым определяются слуюбы и их версии расположены в файле \\ \src{nmap-service-probes}. Для того, чтобы продемонстрировать
его работу создадим собственный минимальный TCP сервер и попробуем добавить соответствующее правило для его распознавания. Исходный код 
сервера представлен в листинге \ref{lst:my-tcp-server}.

\begin{listing}[H]
    \inputminted[linenos=true]{console}{resources/07_my_tcp_server}
    \caption{Исходный код минимального TCP сервера}
    \label{lst:my-tcp-server}
\end{listing}

Механизм работы данной службы тривиален:
\begin{itemize}
    \item В строке $7$ создает сокет для прослушки TCP запросов на порт с номером $6789$.
    \item Внутри бесконечного цикла в строке $9$ происходит ожидание сообщений на прослушиваемый порт.
    \item В строке $14$ в ответ помещается сообщение \src{"Hello, nice to meet you."}.
\end{itemize}
Для запуска данного сервера необходимо поместить данный код в файл \src{TCPServer.java} и разместить его на сканируемой машине. 
Затем данный файл необходимо скомпилировать. Для этого используем команду \src{javac TCPServer.java}. Для запуска полученного таким 
образом \src{.class} файла достаточно выполнить команду \src{java TCPServer}.

Теперь для того, чтобы <<научить>> Nmap распознавать созданную службу необходимо определить какого типа запрос должен быть направлен
в даннуб службу для того, чтобы получить ожидаемый и, по возможности уникальный в пределах остальных известных Nmap сервисов, ответ.
Понятно, что, поскольку ответ, выдаваемый созданным сервисом не зависит от направленного ему запроса, можно использовать любой
TCP запрос, например, запрос с пустым телом. Ответ сервера также не зависит ни от каких условий и будет содержать всегда одну и ту
же строку. Таким, образом, сегмент файла \src{nmap-service-probes}, который отвечает за распознавание написанного TCP сервера,
может выглядеть как показано в листинге \ref{lst:my-tcp-server-probe}.

\begin{listing}[H]
    \inputminted{console}{resources/08_nmap_service_probes}
    \caption{Сегмент файла \src{nmap-service-probes}, определяющий \src{my-tcp}}
    \label{lst:my-tcp-server-probe}
\end{listing}

Для того чтобы проверить корректность определения созданного сервера, воспользуемся уже известной опцией сканирования сервисов с
указанием порта как показано в листинге \ref{lst:scan-port-6789}.

\begin{listing}[H]
    \inputminted{console}{resources/09_scan_port_6789}
    \caption{Пример сканирования порта с номером \src{6789}}
    \label{lst:scan-port-6789}
\end{listing}
\section{Формат вывода}

При формировании отчета Nmap позволяет сохранить его в нескольких форматах, например, XML. Для того, чтобы воспользоваться этой 
опцией необходимо добавить флаг \src{-oX filename} к строке вызова утилиты. Пример использования приведен ниже в листингах 
\ref{lst:scan-port-oX} и \ref{lst:xml-report}

\begin{listing}[H]
    \inputminted{console}{resources/10_scan_port_oX}
    \caption{Пример использования опции \src{-oX}}
    \label{lst:scan-port-oX}
\end{listing}

\begin{listing}[H]
    \inputminted{xml}{resources/11_xml_report}
    \caption{Содержимое файла \src{h100p6789}}
    \label{lst:xml-report}
\end{listing}
\section{Анализ работы Nmap}

Для того, чтобы проанализировать работу Nmap воспользуемся утилитой Wireshark. 

Рассмотрим задачу поиска активных хостов в сети. Для удобства предположим, что заранее известен не очень широкий диапазон потенциально
уязвимых адресов и возможно применение опции \src{-PR}. Таким образом, для решения поставленной задачи можно воспользоваться командой
\begin{center}
    \src{nmap -n -sn 10.0.0.98-102}.
\end{center}
Результаты анализа трафика данной команды с помощью инструмента Wireshark представлены в литиснге \ref{lst:wshark-PR}.

\begin{listing}[H]
    \inputminted[linenos=true]{console}{resources/12_wshark_PR}
    \caption{Анализ трафика для \src{nmap -n -sn 10.0.0.98-102}}
    \label{lst:wshark-PR}
\end{listing}

Как видно из отчета, процесс поиска активных хостов по протоколу ARP достаточно прост: Nmap просто рассылает широковещательный запрос
для того, чтобы узнат, есть ли в сети машина с определенным ip адресом. В случае получения ответа (строка 10) хост считается активным.

Проанализируем теперь сканирование известного сервиса на порту \src{6789}. Для этого воспользуемся командой 
\begin{center}
    \src{nmap -n -sV -p6789 10.0.0.100}.
\end{center}
Результаты анализа приведены в листинге \ref{lst:wshark-P6789}. Провдем подробный разбор:
\begin{itemize}
    \item В строках $2 -- 9$ отражена проверка активности хоста по сценарию, аналогичному \ref{lst:wshark-PR}.  
    \item В строках $10 -- 12$ происходит проверка того, что порт открыт.
    \item В строках $13 -- 15$ происходит процесс тройного рукопожатия и установление соеденения.
    \item В строках $15 -- 23$ происходит обмен сообщениями и завершение соеденения.
\end{itemize}

\begin{listing}[H]
    \inputminted[linenos=true]{console}{resources/13_wshark_P6789}
    \caption{Анализ трафика для \src{nmap -n -sV -p6789 10.0.0.100}}
    \label{lst:wshark-P6789}
\end{listing}
\section{Использование db\_nmap}

Утилита db\_nmap входит в состав фреймворка metasploit. Данный инструмент расширяет возможности Nmap тем, что позволяет сохранять 
результаты сканирования в базу данных. Для того, чтобы начать работу с db\_nmap необходимо сначала запустить сервис базы данных. В 
случае дистрибутива kali linux это будет PostgreSQL. Для запуска сервиса базы данных воспользуемся командой 
\src{service postgresql start}. Для работы непосредственно с db\_nmap теперь необходимо открыть консольный интерфейс для metasploit. 
Для удобной работы с базой данных заведем новый workspace, который будет хранить результаты для сканирования хоста по адресу 
\src{10.0.0.100} как показано в листинге \ref{lst:workspace-a}.

\begin{listing}[H]
    \inputminted{console}{resources/14_workspace_a}
    \caption{Создание нового workspace для работы с db\_nmap}
    \label{lst:workspace-a}
\end{listing}

Синтаксис утилиты db\_nmap повторяет синтаксис вызова Nmap. Для демонстрации работы с db\_nmap проведем сканирование TCP портов хоста
с адресом \src{10.0.0.100} как показано в листинге \ref{lst:db-scan-ports-sV}.

\begin{listing}[H]
    \inputminted{console}{resources/15_db_scan_ports_sV}
    \caption{Пример использования утилиты db\_nmap}
    \label{lst:db-scan-ports-sV}
\end{listing}

Теперь для проверки результатов работы db\_nmap воспользуемся командами \src{hosts} и \src{services} как показано в листингах 
\ref{lst:db-hosts} и \ref{lst:db-services}.

\begin{listing}[H]
    \inputminted{console}{resources/16_db_scan_hosts}
    \caption{Результаты сканирования db\_nmap}
    \label{lst:db-hosts}
\end{listing}

\begin{listing}[H]
    \inputminted{console}{resources/17_db_services}
    \caption{Результаты сканирования db\_nmap}
    \label{lst:db-services}
\end{listing}
\section{Разбора файлов в составе Nmap}

Проведем теперь разбор некоторых файлов в составе утилиты Nmap. Сначала обратимся к файлу nmap-service-probes и рассмотрим несколько 
записей из него.

\begin{description}
    \item \src{Probe TCP NULL q||} \hfill \\
        Указывает использовать в качестве пробы TCP запрос с пустым телом.
    \item \src{totalwaitms 6000} \hfill \\
        Указывает ожидать ответа в течении $6$ти секунд.
    \item \src{match acmp m|^ACMP Server Version ([\w._-]+)\r\n| p/Aagon ACMP Inventory/ v/$1/} \hfill \\
        Данная запись задает строку сравнения для определения ACMP сервиса. В соответствии с регулярным выражением ACMP сервис 
        распознается если строка ответа начинается с подстроки \src{ACMP Server Version} за которой следует последовательность 
        символов, которая будет распознана как версия.
    \item \src{match AndroMouse m|^AMServer$|s p/AndroMouse Android remote mouse server/}
        Данная запись задает строку сравнения для определения сервиса AndroMouse. В соответствии с регулярным выражением данный сервис 
        распознается если строка ответа в точности соответствует \src{AMServer}.
\end{description}

Проведем теперь разбор какого-либо скрипта из состава Nmap. В качестве примера был выбран скрипт \src{vnc-info.nse}. Его исходный 
код прелставлен в листинге \ref{lst:nse}. Разберем работу этого скрипта:
\begin{itemize}
    \item Строки $1 -- 3$ содержат подключение необходимых библиотек.
    \item В строках $5 -- 11$ содержится некоторая метаинформация о скрипте.
    \item В строках $13$ и $14$ содержится описание необходимых локальных переменных и функций.
    \item В строке $19$ создается объект класса типа \src{VNC}. 
    \item В строке $23$ происходит установка соеденения с выбранным портом по указанному адресу.
    \item Затем в строке $26$ роисходит попытка <<авторизации>> клиента. 
    \item Строки с $26$ой по $42$ую отвечают за формирование результата.
\end{itemize}

\begin{listing}[H]
    \inputminted{console}{resources/18_nse}
    \caption{Текст скрипта \sec{vnc-info.nse}}
    \label{lst:nse}
\end{listing}
\section{Заключение}

Утилита Nmap предоставляет широкий функционал по сканированию сети. В рамках данной работы были подробно разобраны некоторые аспекты 
работы с данной утилитой. 

\end{document}