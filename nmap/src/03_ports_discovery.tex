\section{Поиск открытых портов}

При обнаружении открытых портов Nmap также предоставляет несколько опций. По умолчанию используется флаг \src{-sS}, который 
соответствует посылке TCP запроса с флагом \src{SYN}. Преимущество такого подхода заключается в том, что в ходе сканирования не 
устанавливается никаких соединений. Пример сканирования портов утилитой Nmap представлен в листинге \ref{lst:scan-ports-sS}. 

\begin{listing}[H]
    \inputminted{console}{resources/03_scan_ports_sS}
    \caption{Поиск активных портов с использованием Nmap и опции \src{-sS}}
    \label{lst:scan-ports-sS}
\end{listing}

Конечно, использование опции \src{-sS} оправдано только для сканирования портов, на которых запущены службы, способные работать с 
протоколом TCP. Для обнаружения портов, которые, например, настроены на прием UDP пакетов необходимо воспользоваться опцией \src{-sU}
как показано в листинге \ref{lst:scan-ports-sU}.

\begin{listing}[H]
    \inputminted{console}{resources/04_scan_ports_sU}
    \caption{Поиск активных портов с использованием Nmap и опции \src{-sU}}
    \label{lst:scan-ports-sU}
\end{listing}

Еще одной интересной опцией, предоставляемой Nmap для сканирования портов, является флаг \src{-sA}. Функционирование Nmap при 
использовании данной опции заключается в отправке TCP запроса с флагом \src{ACK}. В таком случае утилита не способна определит открыт
или закрыт порт, однако способна определить какие из них отфильтрованы (то есть закрыты брандмауэром (firewall)). Такая функция может
оказаться полезной, например, при проведении аудита безопасни. Пример использования флага \src{-sA} представлен в листинге 
\ref{lst:scan-ports-sA}.

\begin{listing}[H]
    \inputminted{console}{resources/05_scan_ports_sA}
    \caption{Поиск активных портов с использованием Nmap и опции \src{-sA}}
    \label{lst:scan-ports-sA}
\end{listing}