\section{Проверить ЭЦП на стороннем файле}

Для демонстрации проверки подлинности ЭЦП с помощью утилиты gpg были скачаны следующий файлы:
\begin{itemize}
    \item Инсталятор \href{https://files.gpg4win.org/gpg4win-vanilla-2.3.0.exe}{Gpg4win-Vanilla 2.3.0}.
    \item \href{https://files.gpg4win.org/gpg4win-vanilla-2.3.0.exe.sig}{Его цифровая подпись}.
    \item \href{https://ssl.intevation.de/Intevation-Distribution-Key.asc}{Сертификат}, с помощью которого была сделана подпись.
\end{itemize}
Теперь необходимо импортировать полученный сертификат. Для этого предназначена опция \src{--import}. Результат вывода утилиты gpg
при использовании данной опции представлен в листинге \ref{lst:import-certificate}.

\begin{listing}[H]
    \inputminted{console}{resources/06_import_certificate}
    \caption{Вывод утилиты gpg при вызове с опцией \src{--import}}
    \label{lst:import-certificate}
\end{listing}

Для дальнейшего использования импортированного сертификата необходимо подтвердить его полномочия (подписать). Согласно общепринятому 
мнению, данный этап подтверждения является Ахиллесовой пятой gpg, поскольку пользователь системы должен быть на сто процентов
уверен в личности создателя сертификата. Для решения данной проблемы рекомендуется проводить личные встречи с автором сертификата,
однако на практике, как правило, ограничиваются сверкой отпечатков сертификата. Также стоит отметить, что здесь имеет место быть 
концепция web of trust, согласно которой, получаемый сертификат уже может содержать определенное количество подписей от других 
пользователей. Это позволяет судить о степени доверия полученному сертификату по количеству других пользователей, которые подписали
его. Для того, чтобы подтвердить полномочия некоторого сертификата необходимо запустить утилиты gpg с опцией \src{--edit-key}
и затем ввести команду \src{sign}. Пример выполнения данной операции представлен в листинге \ref{lst:sign-certificate}.

\begin{listing}[H]
    \inputminted{console}{resources/07_sign_certificate}
    \caption{Вывод утилиты gpg при вызове с опцией \src{--edit-key}}
    \label{lst:sign-certificate}
\end{listing}

Для проверки электронной подписи теперь достаточно вызывать утилиту gpg с опцией \src{--verify} как показано в листинге 
\ref{lst:verify-sign}.

\begin{listing}[H]
    \inputminted{console}{resources/08_verify_sign}
    \caption{Вывод утилиты gpg при вызове с опцией \src{--verify}}
    \label{lst:verify-sign}
\end{listing}